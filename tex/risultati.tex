\section{Risultati}

\subsection{Punti chiave}
In questa sezione vengono esposti i principali indicatori di qualità 
relativi ai risultati ottenuti in varie casistiche.

Un primo fattore caratterizzante è emerso valutando le 4 preferenze espresse 
tramite i \textit{weak constraints}. La generazione di \textit{Answer Sets} 
tenendo conto di tutti i vincoli risulta molto pesante e il programma non è 
in grado di fornire il risultato ottimo in tempo utile.
Tuttavia, dopo circa 30 secondi di esecuzione (tempo impostato manualmente come limite),
si ottiene un risultato esaustivo in termini di configurazioni dei vari docenti.

\subsection{Preferenze}
È possibile ottenere risultati ottimi in tempi ristretti, eliminando alcune 
preferenze per facilitare il processo di ottimizzazione. In particolare, si raggiungono
prestazioni elevate se non vengono considerati i vincoli relativi ai predicati 
\texttt{ideale/1} e \texttt{non\_ideale/1}.

\subsection{Input ridotto}
I primi test sono stati eseguiti ponendo come obiettivo la generazione di 
configurazioni solo per corsi di studio singoli o a coppie, per verificare 
rapidamente la correttezza dei risultati.
Per fare un esempio pratico, il corso di laurea triennale in informatica (3027)
offre 26 insegnamenti, distribuiti su 20 docenti.
La categoria del corso e le numerosità di immatricolazioni richiedono 
la presenza di almeno 9 garanti, tra cui:
\begin{itemize}
    \item non meno di 5 professori a tempo indeterminato;
    \item non più di 4 ricercatori;
    \item non più di 2 professori a contratto.
\end{itemize}

\begin{table}[!ht]
    \caption{Garanti generati per il corso 3027}
    \centering
    \begin{tabular}{|l|l|l|}
    \hline
        Docente & Matricola & SSD 2015 \\ \hline
        DAL PALU' A. & 6625 & INF/01 \\ \hline
        DE FILIPPIS C. & 34499 & MAT/05 \\ \hline
        BONNICI V. & 34181 & INF/01 \\ \hline
        GUARDASONI C. & 6801 & MAT/08 \\ \hline
        BENINI A. & 26131 & MAT/03 \\ \hline
        DE PIETRI R. & 5536 & FIS/02 \\ \hline
        ZAFFANELLA E. & 5602 & INF/01 \\ \hline
        BERGENTI F. & 204741 & INF/01 \\ \hline
        BAGNARA R. & 5145 & INF/01 \\ \hline
    \end{tabular}
\end{table}

La soluzione ottima, in questo caso, viene generata in meno di 1 secondo, 
quasi in maniera istantanea. 
Nonostante l'esempio sia banale, i vincoli imposti sono stati rispettati 
e la soluzione è facilmente consultabile da parte dell'utente.

In alcuni casi, come per il corso di laurea triennale in chimica (3024), 
il numero di insegnamenti (e di conseguenza il numero di docenti) è molto 
alto, specialmente per i professori a tempo indeterminato. 

In questi casi il programma finisce per valutare una quantità elevatissima 
di modelli, potenzialmente ottimi, tutti tra loro equivalenti. Per questo 
motivo il solver potrebbe evolvere in una situazione di ``stallo'', non 
riuscendo a dimostrare di aver trovato un modello ottimo.

Nel codice \ref{lst:stat_3024} è riportato l'output in questione, da cui è 
possibile notare come la maggior parte del tempo di risoluzione sia impiegato nella 
ricerca (senza successo) di un modello migliore.

Il tempo indicato nel campo \texttt{Unsat} indica, infatti, il tempo trascorso tra 
l'istante in cui è stato trovato l'ultimo modello e la terminazione \cite{gebser2015potassco}.

\begin{lstlisting}[language=bash, captionpos=b, 
    caption={Statistiche clingo per il corso 3024}, 
    label={lst:stat_3024},
    backgroundcolor=\color{lightgray!20},
    basicstyle=\ttfamily\footnotesize]
    Models       : 6
    Optimum      : yes
    Optimization : -1 -9
    Calls        : 1
    Time         : 76.021s 
                    (Solving: 76.01s 
                     1st Model: 0.00s 
                     Unsat: 76.00s)
    CPU Time     : 75.992s
\end{lstlisting}

A questo scopo, è stato introdotto l'utilizzo di un \textit{timer} all'interno 
del codice Python, della durata di 5 secondi.
Nei test eseguiti, questa durata si è rivelata sufficiente per fornire soluzioni 
che rispettassero i vincoli prestabiliti.

