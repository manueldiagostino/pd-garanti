\section{Risultati}

\subsection{Punti chiave}
In questa sezione vengono esposti i principali indicatori di qualità 
relativi ai risultati ottenuti in varie casistiche.

Un primo fattore caratterizzante è emerso valutando le 4 preferenze espresse 
tramite i weak constraint. La generazione di Answer Set,
tenuto conto di tutti i vincoli, risulta molto esosa e il programma non è 
in grado di fornire il risultato ottimo in tempo utile.
Nonostante non sia possibile garantire la produzione della soluzione migliore a causa
dei tempi di esecuzione, le soluzioni fornite dal programma sono da considerare 
corrette ed esaustive, in termini di configurazioni dei docenti.
A questo proposito è stato introdotto un limite di tempo di 5 secondi per 
fermare il risolutore e procedere con l'analisi dell'output.

\subsection{Preferenze}
È possibile ottenere risultati ottimali in tempi brevi (qualche secondo) eliminando alcune 
preferenze, per facilitare il processo di ottimizzazione. In particolare, si raggiungono
prestazioni elevate se vengono tenuti in considerazione solo i vincoli relativi ai 
presidenti e ai professori a tempo indeterminato.
Gli esempi riportati in questa sezione sono stati svolti secondo questa logica.

\subsection{Input ridotto}
I primi test sono stati eseguiti ponendo come obiettivo la generazione di 
configurazioni solo per corsi di studio singoli o a coppie, per verificare 
rapidamente la correttezza dei risultati.
Per fare un esempio pratico, il Corso di Laurea Triennale in Informatica (codice 3027)
offre 26 insegnamenti, distribuiti su 20 docenti.
La categoria del corso e le numerosità di immatricolazioni richiedono 
la presenza di almeno 9 garanti, tra cui:
\begin{itemize}
    \item non meno di 5 professori a tempo indeterminato;
    \item non più di 4 ricercatori;
    \item non più di 2 professori a contratto.
\end{itemize}

\begin{table}[!ht]
    \caption{Garanti Generati per il Corso 3027.}
    \centering
    \begin{tabular}{l c c}
        \toprule
        \textbf{Docente} & \textbf{Matricola} & \textbf{SSD 2015} \\
        \midrule
        Dal Palù A. & 6625 & INF/01 \\
        De Filippis C. & 34499 & MAT/05 \\
        Bonnici V. & 34181 & INF/01 \\
        Guardasoni C. & 6801 & MAT/08 \\
        Benini A. & 26131 & MAT/03 \\
        De Pietri R. & 5536 & FIS/02 \\
        Zaffanella E. & 5602 & INF/01 \\
        Bergenti F. & 204741 & INF/01 \\
        Bagnara R. & 5145 & INF/01 \\
        \bottomrule
    \end{tabular}
    \label{tab:garanti}
\end{table}

La soluzione ottimale, in questo caso, viene generata in meno di 1 secondo. 
Nonostante l'esempio sia banale, i vincoli imposti sono stati rispettati 
e la soluzione è facilmente consultabile da parte dell'utente.

\subsubsection*{Eccezioni di rilievo} Per il Corso di Laurea Triennale in Chimica (codice 3024), 
il numero di insegnamenti (e di conseguenza il numero di docenti) è molto 
alto; in particolare, si riscontra un elevato numero di professori con contratto
a tempo indeterminato. In queste casistiche, il programma finisce per valutare una quantità elevatissima 
di modelli, potenzialmente ottimi, tutti tra loro equivalenti. Per questo 
motivo il solver potrebbe evolvere in una situazione di ``stallo'', non 
riuscendo a dimostrare di aver trovato un modello ottimale. Nel Cod.~\ref{lst:stat_3024} è riportato l'output in questione, da cui è 
possibile notare come la maggior parte del tempo di risoluzione sia impiegato nella 
ricerca (senza successo) di un modello migliore. Il tempo indicato nel campo \texttt{Unsat} indica, infatti, il tempo trascorso tra 
l'istante in cui è stato trovato l'ultimo modello e la terminazione \cite{gebser2015potassco}.
Sono riportati, cambiati di segno, i valori di ottimizzazione raggiunti, ovvero 1 (presidente) e 
9 (professori a tempo indeterminato), che rappresentano i valori migliori per il corso di laurea in questione. 

\begin{lstlisting}[language=bash, captionpos=b, 
    caption={Statistiche clingo per il corso 3024.}, 
    label={lst:stat_3024},
    backgroundcolor=\color{lightgray!20},
    basicstyle=\ttfamily\footnotesize]
    Models       : 6
    Optimum      : yes
    Optimization : -1 -9
    Calls        : 1
    Time         : 76.021s 
                    (Solving: 76.01s 
                     1st Model: 0.00s 
                     Unsat: 76.00s)
    CPU Time     : 75.992s
\end{lstlisting}

Anche per questo motivo, è stato introdotto l'utilizzo di un \textit{timer} all'interno 
del codice Python, della durata di 5 secondi.
Nei test eseguiti, questo limite si è rivelato sufficiente per fornire soluzioni 
che rispettassero i vincoli prestabiliti.

\subsection{Input completo}
La discrepanza tra i due esempi riportati nei test sui corsi singoli mette in 
mostra la complessità del modello, tipica dei problemi affrontati tramite il 
paradigma di programmazione Answer Set.

Un'ulteriore conferma si ottiene con l'esecuzione del programma tenendo in 
considerazione tutti i corsi dell'ateneo. La ricerca del modello ottimo 
si conclude, infatti, quasi in maniera istantanea, contrariamente a quanto 
ci si possa aspettare.
L'ipotesi è che avere più corsi e professori a disposizione renda alcune soluzioni 
meno ottimali in maniera naturale, alleggerendo il carico di lavoro
dell'ottimizzatore.

\begin{lstlisting}[language=bash, captionpos=b, 
    caption={Statistiche clingo per tutti i corsi.}, 
    label={lst:stat_all},
    backgroundcolor=\color{lightgray!20},
    basicstyle=\ttfamily\footnotesize]
    Models       : 196
    Optimum      : yes
    Optimization : -76 -705
    Calls        : 1
    Time         : 0.800s 
                    (Solving: 0.65s 
                     1st Model: 0.36s 
                     Unsat: 0.00s)
    CPU Time     : 0.714s
\end{lstlisting}

\subsubsection*{Massimizzazione delle preferenze}
Utilizzando tutti i vincoli presenti nel file delle preferenze, si ottengono 
configurazioni più specifiche, in base alle proprietà descritte in dettaglio
alla sezione \ref{implementazione}. 
Nel Cod. \ref{lst:stat_all_pref} viene mostrato un esempio, con un limite 
di tempo impostato a 5 secondi. Si notano alcune proprietà a riguardo:
\begin{itemize}
    \item la risoluzione richiede in generale più tempo;
    \item i livelli di ottimizzazione raggiunti relativi a presidenti e 
        professori a tempo indeterminato non cambiano di molto rispetto 
        all'esempio mostrato nel Cod. \ref{lst:stat_all};
		\item il numero di corsi per i quali risulta vero il predicato
		\texttt{ideale/1} (43 corsi) è leggermente inferiore alla sua controparte
		\texttt{non\_ideale/1} (52 corsi).
\end{itemize}

\begin{lstlisting}[language=bash, captionpos=b, 
    caption={Statistiche clingo per tutti i corsi con preferenze massime.}, 
    label={lst:stat_all_pref},
    backgroundcolor=\color{lightgray!20},
    basicstyle=\ttfamily\footnotesize]
    TIME LIMIT   : 1
    Models       : 238+
      Optimum    : unknown
    Optimization : -68 -43 -702 52
    Calls        : 1
    Time         : 5.003s 
                    (Solving: 4.82s 
                     1st Model: 1.38s 
                     Unsat: 0.00s)
    CPU Time     : 4.870s
\end{lstlisting}
