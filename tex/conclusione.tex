\section{Conclusione}

L'esperimento descritto ha dimostrato l'efficacia dell'approccio basato su
Answer Set Programming per affrontare il problema
dell'individuazione automatica dei garanti accademici. Grazie all'uso di
Clingo e Python, è stato possibile implementare un sistema in
grado di analizzare un grande volume di dati e di risolvere un problema
complesso in tempi computazionalmente accettabili.

I risultati ottenuti dimostrano come l'approccio adottato abbia soddisfatto
pienamente i criteri richiesti, garantendo una selezione accurata e conforme
alle specifiche definite. La modellazione logica, combinata con la capacità di
ASP di gestire vincoli complessi e criteri di ottimizzazione, si è dimostrata
uno strumento indispensabile per affrontare questo problema.

Nonostante il successo del sistema, permangono alcune aree di miglioramento. Ad
esempio, l'introduzione della possibilità che un docente possa essere garante
per due corsi di studio con un peso proporzionale (0,5 per ciascuno) potrebbe
rendere il sistema più flessibile, ma al costo di un aumento del tempo di
risoluzione. Inoltre, è necessaria la creazione di una base dati centralizzata,
aggiornata e accessibile in tempo reale; ciò richiederebbe una ristrutturazione
del codice Python esistente ma si avrebbe il vantaggio di operare su dati
completi e aggregati. Sarebbe altresì opportuno sviluppare un'interfaccia utente
per rendere il sistema più accessibile e facile da utilizzare per il personale
accademico.

In conclusione, il progetto rappresenta un primo passo promettente verso
l'automazione intelligente delle attività accademiche. L'adozione futura di
soluzioni avanzate, quali una maggiore integrazione con i sistemi di gestione
universitaria e l'uso di tecnologie complementari, potrebbe ampliare
ulteriormente l'applicabilità e l'impatto di questo lavoro.
