\section{Background}
\subsection{Motivazioni}
Attualmente, l'Università di Parma gestisce l'assegnazione dei docenti in modo
manuale, affrontando il processo in maniera incrementale per ciascun corso di
laurea. Il personale incaricato impiega settimane per ottenere una versione
soddisfacente della distribuzione, basandosi frequentemente su preferenze
informali e criteri non documentati. Questo approccio risulta poco flessibile e
difficilmente adattabile a nuove esigenze. Inoltre, presenta significative
limitazioni, come la difficoltà nel gestire situazioni complesse e l'incapacità
di ottimizzare il processo in tempo reale, rendendo il sistema poco efficiente e
reattivo ai cambiamenti.

\subsection{Answer Set Programming}
L'Answer Set Programming (ASP) è un paradigma di programmazione logica
dichiarativa, particolarmente adatto per risolvere problemi complessi di natura
combinatoria che richiedono soluzioni flessibili e ottimizzate. A differenza
della programmazione imperativa tradizionale, ASP si concentra sulla descrizione
del \textit{cosa} deve essere risolto piuttosto che su	\textit{come} farlo,
utilizzando una forma di logica che rappresenta le conoscenze del problema e le
sue restrizioni (\textit{vincoli}).

ASP si basa sulla teoria degli \textit{answer sets}, ossia un insieme di atomi
letterali consistenti con le regole e i fatti che costituiscono il programma. Il
compito del solver ASP è quello di trovare gli insiemi di valori che risolvono
il sistema di equazioni logiche, fornendo così soluzioni ottimali o ammissibili.

L'uso di questo paradigma di programmazione è particolarmente indicato in ambiti in cui sono presenti vincoli complessi, preferenze multiple e
soluzioni che devono rispettare determinati criteri. ASP permette di modellare
in modo naturale problemi che coinvolgono l'ottimizzazione, la pianificazione, e
la ricerca di soluzioni in scenari combinatori, in cui le variabili e le
relazioni tra esse sono numerose e intricate.

\subsection{Il solver}
\texttt{Clingo} \cite{clingo} è un solver open-source per ASP, sviluppato dal gruppo Potassco.
È uno degli strumenti più potenti e diffusi per risolvere problemi complessi di
ottimizzazione e combinazione, combinando un motore di inferenza logica con
capacità avanzate di ottimizzazione. Nel nostro progetto, è stato integrato
direttamente in Python utilizzando le API Python ufficiali
\cite{clingo_python_api}, le quali consentono di interagire facilmente con il
solver all'interno di ambienti Python. Questa integrazione permette di
automatizzare il processo di invocazione e gestione delle soluzioni, facilitando
l'elaborazione dei dati e l'ottimizzazione delle assegnazioni in tempo reale.

\subsection{Provenienza e contenuto dei dati di input}

I dati di input sono stati forniti dall'U.O. Progettazione Didattica e
Assicurazione della Qualità \cite{unipr_qualita}, in collaborazione con il prof.
A. Dal Palù dell'Università di Parma. Questi includono un insieme di tabelle e
documenti eterogenei, comprendenti le coperture dei corsi per l'anno accademico
corrente, l'elenco del personale docente e le informazioni relative alle
immatricolazioni nei corsi di laurea. L'elaborazione è stata effettuata
utilizzando \texttt{Python} e si è rivelata particolarmente complessa a causa
dell'assenza di una sorgente dati unica e centralizzata.

