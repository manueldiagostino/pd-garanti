\documentclass[draft, journal, onecolumn]{IEEEtran}
\usepackage[italian]{babel}

\usepackage[utf8]{inputenc}
\usepackage[T1]{fontenc}

\setlength{\marginparwidth}{2cm}
\usepackage[italian, textsize=small, textwidth=2cm]{todonotes}

\usepackage{hyperref}
\usepackage{url}

\renewcommand{\IEEEkeywordsname}{Keywords}

\begin{document}

\title{Ottimizzazione Garanti accademici}

\author{\IEEEauthorblockN{1\textsuperscript{st} Given Name Surname}
\IEEEauthorblockA{\textit{dept. name of organization (of Aff.)} \\
\textit{name of organization (of Aff.)}\\
City, Country \\
email address or ORCID}\\
\and
\IEEEauthorblockN{2\textsuperscript{nd} Given Name Surname}
\IEEEauthorblockA{\textit{dept. name of organization (of Aff.)} \\
\textit{name of organization (of Aff.)}\\
City, Country \\
email address or ORCID}
}

\maketitle

\begin{abstract} La gestione dell'assegnazione dei docenti di riferimento o
	garanti in ambito universitario è un compito complesso che coinvolge vari
	fattori organizzativi e un gran numero di dati. L'automazione di questo
	processo può semplificare notevolmente il lavoro amministrativo e migliorare
	l'efficienza operativa delle università. In questo progetto, abbiamo
	sviluppato un sistema basato su \textit{Answer Set Programming} (ASP), un
	paradigma logico adatto alla risoluzione di problemi combinatori complessi.
	Utilizzando \texttt{Python} e \texttt{Clingo}, un potente solver ASP, abbiamo
	progettato una soluzione capace di identificare automaticamente i docenti di
	riferimento, a partire da un insieme di dati specifici. Il sistema sviluppato
	è descritto nei suoi dettagli tecnici, con un focus sulla modellazione logica,
	l'implementazione e l'analisi della complessità computazionale. I risultati
	ottenuti sono discussi insieme alle potenzialità di estensione del sistema.
\end{abstract}

\begin{IEEEkeywords}
Answer Set Programming, Ottimizzazione, Programmazione Dichiarativa, Clingo
\end{IEEEkeywords}

\section{Introduzione} L'identificazione dei docenti di riferimento o garanti
per uno specifico corso di studi rappresenta un processo che annualmente
coinvolge le università italiane. Tale compito può risultare complesso,
specialmente in presenza di strutture organizzative articolate e di grandi
volumi di dati. La necessità di automatizzare e ottimizzare questa ricerca è
quindi cruciale per migliorare l'efficienza delle attività amministrative e
accademiche.

In questo progetto, abbiamo affrontato il problema utilizzando \textit{Answer
Set Programming} (ASP), un paradigma di programmazione logica dichiarativa
particolarmente adatto alla risoluzione di problemi combinatori complessi.
L'implementazione è stata realizzata con l'ausilio di \texttt{Python}
\cite{python} e \texttt{Clingo} \cite{clingo}, un solver open source ASP che
combina il modello di programmazione logica con strumenti di ottimizzazione
efficienti.

L'obiettivo principale è stato quello di progettare e implementare un sistema
che, a partire da un insieme di dati resi disponibili dagli uffici di
competenza, consenta di individuare in maniera automatica i docenti di
riferimento o garanti in base a criteri specifici. La relazione descrive le fasi
del lavoro, dall'analisi dei requisiti del problema alla modellazione logica,
presentando i dettagli dell'implementazione concreta e un'analisi sulla sua
complessità computazionale. Infine, vengono discussi i risultati ottenuti e le
possibili estensioni del progetto.


\section{Background}
Richiami teorici ad ASP, aspetti tecnici e specifiche.

\section{Modellazione del problema}
\subsection{Esempio giocattolo}
\subsection{Implementazione}
\subsection{Complessità computazionale}

\section{Risultati}

\section{Conclusione}

\bibliographystyle{IEEEtran}
\bibliography{main}

\end{document}
