\documentclass[draft,journal]{IEEEtran}
\usepackage[italian]{babel}

\usepackage[utf8]{inputenc}
\usepackage[T1]{fontenc}

\setlength{\marginparwidth}{2cm}
\usepackage[italian, textsize=small, textwidth=2cm]{todonotes}

\usepackage{hyperref}
\usepackage{url}

\renewcommand{\IEEEkeywordsname}{Keywords}

\begin{document}

\title{Ottimizzazione Garanti accademici}

\author{\IEEEauthorblockN{1\textsuperscript{st} Given Name Surname}
\IEEEauthorblockA{\textit{dept. name of organization (of Aff.)} \\
\textit{name of organization (of Aff.)}\\
City, Country \\
email address or ORCID}\\
\and
\IEEEauthorblockN{2\textsuperscript{nd} Given Name Surname}
\IEEEauthorblockA{\textit{dept. name of organization (of Aff.)} \\
\textit{name of organization (of Aff.)}\\
City, Country \\
email address or ORCID}
}

\maketitle

\begin{abstract} La gestione dell'assegnazione dei docenti di riferimento o
	garanti in ambito universitario è un compito complesso che coinvolge vari
	fattori organizzativi e un gran numero di dati. L'automazione di questo
	processo può semplificare notevolmente il lavoro amministrativo e migliorare
	l'efficienza operativa delle università. In questo progetto, abbiamo
	sviluppato un sistema basato su \textit{Answer Set Programming} (ASP), un
	paradigma logico adatto alla risoluzione di problemi combinatori complessi.
	Utilizzando \texttt{Python} e \texttt{Clingo}, un potente solver ASP, abbiamo
	progettato una soluzione capace di identificare automaticamente i docenti di
	riferimento, a partire da un insieme di dati specifici. Il sistema sviluppato
	è descritto nei suoi dettagli tecnici, con un focus sulla modellazione logica,
	l'implementazione e l'analisi della complessità computazionale. I risultati
	ottenuti sono discussi insieme alle potenzialità di estensione del sistema.
\end{abstract}

\begin{IEEEkeywords}
Answer Set Programming, Ottimizzazione, Programmazione Dichiarativa, Clingo
\end{IEEEkeywords}

\section{Introduzione} L'identificazione dei docenti di riferimento o garanti
per uno specifico corso di studi rappresenta un processo che annualmente
coinvolge le università italiane. Tale compito può risultare complesso,
specialmente in presenza di strutture organizzative articolate e di grandi
volumi di dati. La necessità di automatizzare e ottimizzare questa ricerca è
quindi cruciale per migliorare l'efficienza delle attività amministrative e
accademiche.

In questo progetto, abbiamo affrontato il problema utilizzando \textit{Answer
Set Programming} (ASP), un paradigma di programmazione logica dichiarativa
particolarmente adatto alla risoluzione di problemi combinatori complessi.
L'implementazione è stata realizzata con l'ausilio di \texttt{Python}
\cite{python} e \texttt{Clingo} \cite{clingo}, un solver open source ASP che
combina il modello di programmazione logica con strumenti di ottimizzazione
efficienti.

L'obiettivo principale è stato quello di progettare e implementare un sistema
che, a partire da un insieme di dati resi disponibili dagli uffici di
competenza, consenta di individuare in maniera automatica i docenti di
riferimento o garanti in base a criteri specifici. La relazione descrive le fasi
del lavoro, dall'analisi dei requisiti del problema alla modellazione logica,
presentando i dettagli dell'implementazione concreta e un'analisi sulla sua
complessità computazionale. Infine, vengono discussi i risultati ottenuti e le
possibili estensioni del progetto.


\section{Background}
\subsection{Motivazioni}
Attualmente, l'Università di Parma gestisce l'assegnazione dei docenti in modo
manuale, affrontando il processo in maniera incrementale per ciascun corso di
laurea. Il personale incaricato impiega settimane per ottenere una versione
soddisfacente della distribuzione, basandosi frequentemente su preferenze
informali e criteri non documentati. Questo approccio risulta poco flessibile e
difficilmente adattabile a nuove esigenze. Inoltre, presenta significative
limitazioni, come la difficoltà nel gestire situazioni complesse e l'incapacità
di ottimizzare il processo in tempo reale, rendendo il sistema poco efficiente e
reattivo ai cambiamenti.

\subsection{Answer Set Programming}
L'Answer Set Programming (ASP) è un paradigma di programmazione logica
dichiarativa, particolarmente adatto per risolvere problemi complessi di natura
combinatoria che richiedono soluzioni flessibili e ottimizzate. A differenza
della programmazione imperativa tradizionale, ASP si concentra sulla descrizione
del \textit{cosa} deve essere risolto piuttosto che su	\textit{come} farlo,
utilizzando una forma di logica che rappresenta le conoscenze del problema e le
sue restrizioni (\textit{vincoli}).

ASP si basa sulla teoria degli \textit{answer sets}, ossia un insieme di atomi
letterali consistenti con le regole e i fatti che costituiscono il programma. Il
compito del solver ASP è quello di trovare gli insiemi di valori che risolvono
il sistema di equazioni logiche, fornendo così soluzioni ottimali o ammissibili.

L'uso di questo paradigma di programmazione è particolarmente indicato in ambiti
come il nostro, in cui sono presenti vincoli complessi, preferenze multiple e
soluzioni che devono rispettare determinati criteri. ASP permette di modellare
in modo naturale problemi che coinvolgono l'ottimizzazione, la pianificazione, e
la ricerca di soluzioni in scenari combinatori, in cui le variabili e le
relazioni tra esse sono numerose e intricate.

\subsection{\texttt{Clingo}}
\texttt{Clingo} è un solver open-source per ASP, sviluppato dal gruppo Potassco.
È uno degli strumenti più potenti e diffusi per risolvere problemi complessi di
ottimizzazione e combinazione, combinando un motore di inferenza logica con
capacità avanzate di ottimizzazione. Nel nostro progetto, è stato integrato
direttamente in Python utilizzando le API Python ufficiali
\cite{clingo_python_api}, le quali consentono di interagire facilmente con il
solver all'interno di ambienti Python. Questa integrazione permette di
automatizzare il processo di invocazione e gestione delle soluzioni, facilitando
l'elaborazione dei dati e l'ottimizzazione delle assegnazioni in tempo reale.

\subsection{Provenienza e contenuto dei dati di input}
AAA

\section{Modellazione del problema}
\subsection{Esempio giocattolo}
\subsection{Implementazione}
\subsection{Complessità computazionale}

\section{Risultati}

\section{Conclusione}

\bibliographystyle{IEEEtran}
\bibliography{main}

\end{document}
