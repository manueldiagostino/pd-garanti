\section{Introduzione} L'identificazione dei docenti di riferimento o garanti
per uno specifico corso di studi rappresenta un processo che annualmente
coinvolge le università italiane. Tale compito può risultare complesso,
specialmente in presenza di strutture organizzative articolate e di grandi
volumi di dati. La necessità di automatizzare e ottimizzare questa ricerca è
quindi cruciale per migliorare l'efficienza delle attività amministrative e
accademiche.

In questo progetto, il problema è stato affrontato utilizzando \textit{Answer
Set Programming} (ASP), un paradigma di programmazione logica dichiarativa
particolarmente adatto alla risoluzione di problemi combinatori complessi.
L'implementazione è stata realizzata con l'ausilio di \texttt{Python}
\cite{python} e \texttt{Clingo} \cite{clingo}, un solver open source ASP che
combina il modello di programmazione logica con strumenti di ottimizzazione
efficienti.

L'obiettivo principale è stato quello di progettare e implementare un sistema
che, a partire da un insieme di dati resi disponibili dagli uffici di
competenza, consenta di individuare in maniera automatica i docenti di
riferimento o garanti in base a criteri specifici. La relazione descrive le fasi
del lavoro, dall'analisi dei requisiti del problema alla modellazione logica,
presentando i dettagli dell'implementazione concreta e un'analisi sulla sua
complessità computazionale. Infine, vengono discussi i risultati ottenuti e le
possibili estensioni del progetto.
